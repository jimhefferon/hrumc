\abstract{% title: The use of the $f(x)=x^2$ function
    Comparing Methods for Constructing Confidence Intervals Using Simulations in R
  }{% authors: Steve Jones (UBC), Mark Smith (BCU)
    Nanjiang Liu, St. Lawrence University
  }{% level: 1 or 2 (put it next to the closing curly)
    1}{
  % abstract; you can use any AMS math
A confidence interval (CI) is a pair of numbers, based on sample data, designed to capture the value of some population parameter. There are several different methods for constructing CI. For example, we can find a CI for proportion using a formula based on normal distribution, a bootstrap distribution of simulation in proportions, a ``plus 4'' adjustments to proportion, or Bayesian credible interval. We use R simulations to generate many samples from different populations and then compare the coverage rates and widths of the intervals with each method. We vary the population proportion and sample sizes to explore which methods might work best in different situations.
}

% To save a new abstract copy this stub to a new file fff.tex and fill in the 
% fields.
%
% Then, in the hrumcxxxx.tex file, in a session, enter something like 
%  \at{\Ia}{fff} 
% where Ia means session I, first time slot
% and fff is the name of this .tex file 